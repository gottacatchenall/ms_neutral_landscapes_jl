%!TEX TS-program = xelatex
\documentclass[10pt,oneside]{article}
\usepackage[fontsize=9pt]{scrextend}

\usepackage[english]{babel}

\usepackage{amsmath,amssymb,amsfonts}
\usepackage[utf8]{inputenc}
\usepackage[T1]{fontenc}
\usepackage{stix2}
\usepackage[scaled]{helvet}
\usepackage[scaled]{inconsolata}

\usepackage{lastpage}

\usepackage{setspace}

\usepackage{ccicons}

\usepackage[hang,flushmargin]{footmisc}

\usepackage{geometry}

\setlength{\parindent}{0pt}
\setlength{\parskip}{6pt plus 2pt minus 1pt}

\usepackage{fancyhdr}
\renewcommand{\headrulewidth}{0pt}\providecommand{\tightlist}{%
  \setlength{\itemsep}{0pt}\setlength{\parskip}{0pt}}

\makeatletter
\newcounter{tableno}
\newenvironment{tablenos:no-prefix-table-caption}{
  \caption@ifcompatibility{}{
    \let\oldthetable\thetable
    \let\oldtheHtable\theHtable
    \renewcommand{\thetable}{tableno:\thetableno}
    \renewcommand{\theHtable}{tableno:\thetableno}
    \stepcounter{tableno}
    \captionsetup{labelformat=empty}
  }
}{
  \caption@ifcompatibility{}{
    \captionsetup{labelformat=default}
    \let\thetable\oldthetable
    \let\theHtable\oldtheHtable
    \addtocounter{table}{-1}
  }
}
\makeatother

\usepackage{array}
\newcommand{\PreserveBackslash}[1]{\let\temp=\\#1\let\\=\temp}
\let\PBS=\PreserveBackslash

\usepackage[breaklinks=true]{hyperref}
\hypersetup{colorlinks,%
citecolor=blue,%
filecolor=blue,%
linkcolor=blue,%
urlcolor=blue}
\usepackage{url}

\usepackage{caption}
\setcounter{secnumdepth}{0}
\usepackage{cleveref}

\usepackage{graphicx}
\makeatletter
\def\maxwidth{\ifdim\Gin@nat@width>\linewidth\linewidth
\else\Gin@nat@width\fi}
\makeatother
\let\Oldincludegraphics\includegraphics
\renewcommand{\includegraphics}[1]{\Oldincludegraphics[width=\maxwidth]{#1}}

\usepackage{longtable}
\usepackage{booktabs}

\usepackage{color}
\usepackage{fancyvrb}
\newcommand{\VerbBar}{|}
\newcommand{\VERB}{\Verb[commandchars=\\\{\}]}
\DefineVerbatimEnvironment{Highlighting}{Verbatim}{commandchars=\\\{\}}
% Add ',fontsize=\small' for more characters per line
\usepackage{framed}
\definecolor{shadecolor}{RGB}{248,248,248}
\newenvironment{Shaded}{\begin{snugshade}}{\end{snugshade}}
\newcommand{\KeywordTok}[1]{\textcolor[rgb]{0.13,0.29,0.53}{\textbf{#1}}}
\newcommand{\DataTypeTok}[1]{\textcolor[rgb]{0.13,0.29,0.53}{#1}}
\newcommand{\DecValTok}[1]{\textcolor[rgb]{0.00,0.00,0.81}{#1}}
\newcommand{\BaseNTok}[1]{\textcolor[rgb]{0.00,0.00,0.81}{#1}}
\newcommand{\FloatTok}[1]{\textcolor[rgb]{0.00,0.00,0.81}{#1}}
\newcommand{\ConstantTok}[1]{\textcolor[rgb]{0.00,0.00,0.00}{#1}}
\newcommand{\CharTok}[1]{\textcolor[rgb]{0.31,0.60,0.02}{#1}}
\newcommand{\SpecialCharTok}[1]{\textcolor[rgb]{0.00,0.00,0.00}{#1}}
\newcommand{\StringTok}[1]{\textcolor[rgb]{0.31,0.60,0.02}{#1}}
\newcommand{\VerbatimStringTok}[1]{\textcolor[rgb]{0.31,0.60,0.02}{#1}}
\newcommand{\SpecialStringTok}[1]{\textcolor[rgb]{0.31,0.60,0.02}{#1}}
\newcommand{\ImportTok}[1]{#1}
\newcommand{\CommentTok}[1]{\textcolor[rgb]{0.56,0.35,0.01}{\textit{#1}}}
\newcommand{\DocumentationTok}[1]{\textcolor[rgb]{0.56,0.35,0.01}{\textbf{\textit{#1}}}}
\newcommand{\AnnotationTok}[1]{\textcolor[rgb]{0.56,0.35,0.01}{\textbf{\textit{#1}}}}
\newcommand{\CommentVarTok}[1]{\textcolor[rgb]{0.56,0.35,0.01}{\textbf{\textit{#1}}}}
\newcommand{\OtherTok}[1]{\textcolor[rgb]{0.56,0.35,0.01}{#1}}
\newcommand{\FunctionTok}[1]{\textcolor[rgb]{0.00,0.00,0.00}{#1}}
\newcommand{\VariableTok}[1]{\textcolor[rgb]{0.00,0.00,0.00}{#1}}
\newcommand{\ControlFlowTok}[1]{\textcolor[rgb]{0.13,0.29,0.53}{\textbf{#1}}}
\newcommand{\OperatorTok}[1]{\textcolor[rgb]{0.81,0.36,0.00}{\textbf{#1}}}
\newcommand{\BuiltInTok}[1]{#1}
\newcommand{\ExtensionTok}[1]{#1}
\newcommand{\PreprocessorTok}[1]{\textcolor[rgb]{0.56,0.35,0.01}{\textit{#1}}}
\newcommand{\AttributeTok}[1]{\textcolor[rgb]{0.77,0.63,0.00}{#1}}
\newcommand{\RegionMarkerTok}[1]{#1}
\newcommand{\InformationTok}[1]{\textcolor[rgb]{0.56,0.35,0.01}{\textbf{\textit{#1}}}}
\newcommand{\WarningTok}[1]{\textcolor[rgb]{0.56,0.35,0.01}{\textbf{\textit{#1}}}}
\newcommand{\AlertTok}[1]{\textcolor[rgb]{0.94,0.16,0.16}{#1}}
\newcommand{\ErrorTok}[1]{\textcolor[rgb]{0.64,0.00,0.00}{\textbf{#1}}}
\newcommand{\NormalTok}[1]{#1}

\newlength{\cslhangindent}
\setlength{\cslhangindent}{1.5em}
\newlength{\csllabelwidth}
\setlength{\csllabelwidth}{3em}
\newenvironment{CSLReferences}[3] % #1 hanging-ident, #2 entry spacing
 {% don't indent paragraphs
  \setlength{\parindent}{0pt}
  % turn on hanging indent if param 1 is 1
  \ifodd #1 \everypar{\setlength{\hangindent}{\cslhangindent}}\ignorespaces\fi
  % set entry spacing
  \ifnum #2 > 0
  \setlength{\parskip}{#2\baselineskip}
  \fi
 }%
 {}
\usepackage{calc} % for \widthof, \maxof
\newcommand{\CSLBlock}[1]{#1\hfill\break}
\newcommand{\CSLLeftMargin}[1]{\parbox[t]{\maxof{\widthof{#1}}{\csllabelwidth}}{#1}}
\newcommand{\CSLRightInline}[1]{\parbox[t]{\linewidth}{#1}}
\newcommand{\CSLIndent}[1]{\hspace{\cslhangindent}#1}\usepackage[table,dvipsnames]{xcolor}

\geometry{includemp,
            letterpaper,
            top=2.4cm,
            bottom=2.4cm,
            left=1.0cm,
            right=1.0cm,
            marginparwidth=5cm,
            marginparsep=1.0cm}

\usepackage[singlelinecheck=off]{caption}

\captionsetup{
  font={small},
  labelfont={bf},
  format=plain,
  labelsep=quad
}

\usepackage{floatrow}

\floatsetup[figure]{margins=hangright,
              facing=no,
              capposition=beside,
              capbesideposition={center,outside},
              floatwidth=\textwidth}

% \floatsetup[table]{margins=hangright,
%              facing=no,
%              capposition=beside,
%              capbesideposition={center,outside},
%              floatwidth=\textwidth}

\pagestyle{plain}

\setcounter{secnumdepth}{5}

\usepackage{titlesec}

\titleformat{\section}[block]
{\normalfont\large\sffamily}
{\thesection}{.5em}{\titlerule\\[.8ex]\bfseries}

\titleformat{\subsection}[runin]
{\normalfont\fontseries{b}\selectfont\filright\sffamily}
{\thesubsection.}{.5em}{}

\titleformat{\subsubsection}[runin]
{\normalfont\itshape\rmfamily\bfseries}{\thesubsubsection}{1em}{}

\fancypagestyle{firstpage}
{
   \fancyhf{}
   \renewcommand{\headrulewidth}{0pt}
   \fancyfoot[R]{\footnotesize\ccby}
   \fancyfoot[L]{\footnotesize\sffamily\today}
}

\fancypagestyle{normal}
{
  \fancyhf{}
  \fancyfoot[R]{\footnotesize\sffamily\thepage\ of \pageref*{LastPage}}
}

\usepackage{tikz}
\begin{document}
\pagestyle{normal}
\thispagestyle{firstpage}

\newcommand{\colorRule}[3][black]{\textcolor[HTML]{#1}{\rule{#2}{#3}}}

\noindent {\LARGE \textbf{\textsf{NeutralLandscapes.jl: a library for
efficient generation of neutral landscapes with temporal change}}}

\medskip
\begin{flushleft}
{\small
%
\href{https://orcid.org/0000-0002-6506-6487}{Michael D.\,Catchen}%
%
\,\textsuperscript{1,2}
\vskip 1em
\textsuperscript{1}\,McGill University; \textsuperscript{2}\,Québec
Centre for Biodiversity Sciences\\
\vskip 1em
\textbf{Correspondance to:}\\
Michael D. Catchen --- \texttt{michael.catchen@mail.mcgill.ca}\\
}
\end{flushleft}

\vskip 2em
\makebox[0pt][l]{\colorRule[CCCCCC]{2.0\textwidth}{0.5pt}}
\vskip 2em
\noindent

\marginpar{\vskip 1em\flushright
{\small{\bfseries Keywords}:\par
landscape ecology\\spatial ecology\\neutral landscapes\\simulation\\}
}


Soon to be a paper, maybe. TK authors, MKB,VB,RS,TP




\vskip 2em
\makebox[0pt][l]{\colorRule[CCCCCC]{2.0\textwidth}{0.5pt}}
\vskip 2em

\hypertarget{introduction}{%
\section{Introduction}\label{introduction}}

Neutral landscapes are increasingly used in ecological and evolutionary
studies to provide a null expectation of the variance of a given metric
over space.

Wide range of disciplines: landscape genetics to biogeography.

As biodiversity science becomes increasingly concerned with temporal
change and its consequences, its clear there is a gap generating neutral
landscapes that change over time. In this ms we present how
\texttt{NeutralLandscapes.jl} is orders of magnitudes faster than
packages \texttt{nlmpy} (in python) or \texttt{NLMR} (in R). We then
present a novel method for generating landscape change with prescribed
levels of spatial and temporal autocorrelation, and demonstrate that it
works

\hypertarget{software-overview}{%
\section{Software Overview}\label{software-overview}}

This software can generate neutral landscapes using several methods,
enables masking and works with other julia packages.

fig.~\ref{fig:allmethods} shows a replica of Figure 1 from
(\textbf{nlmpycite?})

Table of methods.

\begin{figure}
\hypertarget{fig:allmethods}{%
\centering
\includegraphics{./figures/figure1.png}
\caption{Recreation of the figure in \texttt{nlmpy} paper and the
source}\label{fig:allmethods}
}
\end{figure}

\begin{verbatim}
using NeutralLandscapes, Plots
siz = 50, 50

Fig1a = rand(NoGradient(), siz) # Random NLM
Fig1b = rand(PlanarGradient(), siz) # Planar gradient NLM
Fig1c = rand(EdgeGradient(), siz) # Edge gradient NLM
Fig1d = falses(siz) 
Fig1d[10:25, 10:25] .= true # Mask example
Fig1e = rand(DistanceGradient(findall(vec(Fig1d))), siz) # Mask example
Fig1f = rand(MidpointDisplacement(0.75), siz) # Mask example
Fig1g = rand(RectangularCluster(4, 8), siz)
Fig1h = rand(NearestNeighborElement(200), siz)
Fig1i = rand(NearestNeighborCluster(0.4), siz)
Fig1j = blend([Fig1f, Fig1c])
Fig1k = blend(Fig1h, Fig1e, 1.5)
Fig1l = classify(Fig1i, ones(4))
Fig1m = classify(Fig1a, [1-0.5, 0.5])
Fig1n = classify(Fig1g, [1-0.75, 0.75])
Fig1o = classify(Fig1f, ones(3))
Fig1p = classify(Fig1f, ones(3), Fig1d)
Fig1q = rand(PlanarGradient(90), siz, mask = Fig1n .== 2) #TODO mask as keyword + should mask be matrix or vec or both? (Fig1e)
Fig1r = ifelse.(Fig1o .== 2, Fig1m .+ 2, Fig1o)
Fig1s = rotr90(Fig1l)
Fig1t = Fig1o'

class = cgrad(:Set3_4, 4, categorical = true)
c2, c3, c4 = class[1:2], class[1:3], class[1:4]

gr(color = :fire, ticks = false, framestyle = :box, dpi=500, colorbar = false)
plot(
    heatmap(Fig1a),         heatmap(Fig1b),         heatmap(Fig1c),         heatmap(Fig1d, c = c2), heatmap(Fig1e),
    heatmap(Fig1f),         heatmap(Fig1g),         heatmap(Fig1h),         heatmap(Fig1i),         heatmap(Fig1j), 
    heatmap(Fig1k),         heatmap(Fig1l, c = c4), heatmap(Fig1m, c = c2), heatmap(Fig1n, c = c2), heatmap(Fig1o, c = c3),
    heatmap(Fig1p, c = c4), heatmap(Fig1q),         heatmap(Fig1r, c = c4), heatmap(Fig1s, c = c4), heatmap(Fig1t, c = c3),
    layout = (4,5), size = (1600, 1270)
)

savefig("./figures/figure1.png", )
\end{verbatim}

\hypertarget{interoperability}{%
\subsection{Interoperability}\label{interoperability}}

Ease of use with other julia packages

Mask of neutral variable masked across quebec in 3 lines.

\hypertarget{benchmark-comparison-to-nlmpy-and-nlmr}{%
\section{\texorpdfstring{Benchmark comparison to \texttt{nlmpy} and
\texttt{NLMR}}{Benchmark comparison to nlmpy and NLMR}}\label{benchmark-comparison-to-nlmpy-and-nlmr}}

It's fast. As the scale and resolution of raster data increases, neutral
models must be able to scale to match those data dimensions.

Here we provide two benchmark tests.

First a comparison of the speed variety of methods from each
\texttt{NeutralLandscapes.jl}, \texttt{NLMR}, and \texttt{nlmpy}.

Second we compare these performance of each of these software packages
as rasters become larger. We show that \texttt{Julia} even outperforms
the \texttt{NLMR} via C++ implemention of a particularly slow neutral
landscape method (midpoint displacement).

\textbf{Fig 2}: Benchmark comparison of selected methods in each of the
three languages

\textbf{Fig 3} scale comparison

\begin{figure}
\centering
\includegraphics{}
\caption{Comparison of speed of generating a midpoint displacement
neutral landscape (y-axis) against raster size (measured as length of
the size of a square raster, x-axis)}
\end{figure}

\hypertarget{generating-dynamic-neutral-landscapes}{%
\section{Generating dynamic neutral
landscapes}\label{generating-dynamic-neutral-landscapes}}

We implement methods for generating change that are temporally
autocorrelated, spatially autocorrelated, or both.

\(M_t = f(M_{t-1})\)

\hypertarget{discussion}{%
\section{Discussion}\label{discussion}}

\hypertarget{references}{%
\section{References}\label{references}}

\end{document}
